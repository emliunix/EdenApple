\section*{The little schemer\cite{friedman96little} foreword and preface}
\addcontentsline{toc}{section}{The little schemer foreword and preface}

这是《The Little LISPer》一书的第二与第三版的前言。我们获得了作者的允许并将其重新打印于此。

\begin{quote}
\itshape
注:这段翻译未获得几位老人家的授权。
\end{quote}

1967年的时候我参加了一门摄影入门课程。大部分学生(包括我)参加这门课程是为了学习如何变得富有创造力 --- 可以拍出我欣赏的艺术家比如Edward Weston那样水平的照片。第一天老师耐心的向我们解释了一长串列表他这个学期要教给我们的技术。一个关键点是Ansel Adams的``Zone System''。这个技术能够预览一张照片的打印值(最终成品的暗度),以及这个值是如何从场景中的光强推导出来的。为了学习这个技巧,我们必须要学习如何使用曝光表来测量光强以及如何使用曝光时间和显影时间去控制图像中的暗度和对比度。而掌握这些技术有需要掌握更低级别的技巧,比如放胶卷,显影和打印,混合化学制品。一个人必须学会仪式化处理这些敏感材料的过程才能够在多年的实践之后获得持续的产出。在第一次实验课中,收获就是发现显影剂是滑滑的,定影剂闻起来太糟糕了。

那么什么又是创造性创作呢?为了变得有创造性,一个人首先要获得对介质的掌控力。一个人要是连照片都拍不出来又怎么能去思考如何组织一张好的照片呢。在工程实践中,就像其他创造性艺术一样,我们必须先学会分析才能够进行开发工作。一个人造桥,如果没有钢筋,泥土的知识,以及大量的数学技巧来计算桥体结构的相关属性,又怎么能够建造出漂亮又实用的桥梁呢?同样的,一个人要是在如何``预览''自己写的函数产生的计算过程的理解上不够坚深,那又怎么能构建出漂亮的计算机系统呢?

一些摄影师选择使用黑白8x10的板子,也有一些摄影师选择使用35mm胶片。每一个都有相应的优点和缺点。就像摄影一样,编程也需要选择一种介质。Lisp这种介质是那些喜爱自由风格和灵活性的人们的选择。Lisp最初是递归理论和符号代数的理论便车。现如今已经发展成了一个特别强大而又灵活的软件开发工具族系。他为软件系统的快速原型提供了一站式服务。和其他语言一样,Lisp可以看作是胶水,可以粘合数量巨大的封装好的库。而这些库来自于用户社区的各位成员。在Lisp中,过程是第一个公民,可以作为参数传递,作为值返回,还能存储在数据结构中。这种灵活性很具有价值,但最重要的是,这提供了一种机制,一种形式化,命名并存储常用操作(\textbf{idiom})的机制 --- 在工程设计中非常核心的一些普适使用模式。而且,Lisp程序可以很容易的操作Lisp程序本身的形态 --- 这个特性促使了大量的程序组合与分析工具的开发,比如交叉引用。

《The Little LISPer》 以一种独特的方式展开描述了Lisp中创造性编程的技巧。他以大量的智慧将许多钻头和经验打包,让我们愉悦地学习构造递归函数,操作递归数据结构的技巧。对Lisp编程的学生来说,《The Little LISPer》的作用就好比是Hanon或Czerny的练习集对钢琴学生的作用。

{
% \setlength{\parindent}{0pt}
\vspace{30pt}
\begin{flushright}
Gerald J. Sussman\\
马萨诸塞州,剑桥
\end{flushright}
}
