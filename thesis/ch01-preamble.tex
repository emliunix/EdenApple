\chapter{绪论}
\label{ch:pre}

这个章节简单描述LISP解释器的基本结构以及本论文所描述的解释器的具体实现选择。

\section{基本结构}

\section{EdenApple的结构}

\section{Monadic Parser Combinator}

在Haskell中,对Monad这个typeclass的定义如下:

\begin{minted}
  [mathescape,
  linenos,
  numbersep=5pt,
  gobble=2,
  frame=lines,
  framesep=2mm]{haskell}

  class Applicative m => Monad (m :: * -> *) where
    (>>=) :: m a -> (a -> m b) -> m b
    (>>) :: m a -> m b -> m b
    return :: a -> m a
    fail :: String -> m a
    {-# MINIMAL (>>=) #-}
    -- Defined in ‘GHC.Base’
  
  class Functor f => Applicative (f :: * -> *) where
    pure :: a -> f a
    (<*>) :: f (a -> b) -> f a -> f b
    (*>) :: f a -> f b -> f b
    (<*) :: f a -> f b -> f a
    {-# MINIMAL pure, (<*>) #-}
    -- Defined in ‘GHC.Base’
  
  class Functor (f :: * -> *) where
    fmap :: (a -> b) -> f a -> f b
    (<$) :: a -> f b -> f a
    {-# MINIMAL fmap #-}
    -- Defined in ‘GHC.Base’
\end{minted}