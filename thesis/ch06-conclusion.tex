\chapter{总结}

与C,Java这一类语言相比较,scheme语言具有动态作用域,一等函数(闭包)。这使得程序编写时数据和操作的组织更加灵活。call/cc功能更是为编写组织复杂控制结构提供了强有力的武器。在EdenApple中,实现了一等函数,call/cc,尾递归优化这三个基本的特性。

然而EdenApple任然存在着无法忽视的缺陷。

因为一等函数与词法作用域的共同作用(闭包),EdenApple中的环境使用最为直接的堆模型实现。又因为call/cc需要保存控制栈的信息,控制栈也采用堆的方式实现。

Heap based model由于其控制结构都是基于堆实现的,需要额外的指针以及引用操作。因此执行效率上较为低下。在论文\cite{dybvig87timpl}中,给出了更加高效的模型。

由于此处没有实现宏,因此缺乏直接定义\texttt{and or}为宏的能力。And or不同于其他操作符的地方在于,一般操作符先对操作数求值,然后才是操作符的操作。而若是and的左操作符为假,则该操作右边的部分不会执行,这种区别在语句有副作用以及可能产生递归时非常显著。比如下面的例子中,若and是函数,则会无限递归下去。

\begin{code}
\begin{minted}{scheme}
(letrec ([recnoop (lambda (n)
                    (and (> n 0) (recnoop (- n 1))))])
  (recnoop 3))
\end{minted}
\caption{and操作示例}
\end{code}