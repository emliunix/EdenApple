\chapter*{总结}
\thispagestyle{hubu@thesis}
\addcontentsline{toc}{chapter}{总结}

在本论文中,EdenApple实现了一等函数,call/cc,尾递归优化这三个基本的特性。

与C,Java这一类语言相比较,EdenApple具有动态作用域,一等函数(闭包)。这使得程序编写时数据和操作的组织更加灵活统一。call/cc功能更是为编写组织复杂控制结构提供了强有力的武器。

然而EdenApple任然存在着无法忽视的缺陷。

因为一等函数与词法作用域的共同作用(闭包),EdenApple中的环境使用最为直接的堆模型实现。又因为call/cc需要保存控制栈的信息,控制栈也采用堆的方式实现。

Heap based model由于其控制结构都是基于堆实现的,需要额外的指针以及引用操作。因此执行效率上较为低下。在论文\cite{dybvig87timpl}中,给出了更加高效的模型。

由于此处没有实现宏,因此缺乏直接定义\texttt{and or}为宏的能力。\texttt{and or}不同于其他操作符的地方在于,一般操作符先对操作数求值,然后才是操作符的操作。而若是and的左操作符为假,则该操作右边的部分不会执行,这种区别在语句有副作用以及可能产生递归时非常显著。因此在某些情况下,若and是函数,则会无限递归下去。

在\textit{Essentials of Programming Language\cite{friedman2001eopl}}的Foreword中,有这么一个描述,通常在一个大型软件系统中,都需要一种类似解释器的机制用以灵活组合系统中的各个功能模块。如果将每一个功能模块看作解决问题的利刃,那么这种组合的能力便是挥动利刃的手臂。而解释器便是这么一种提供强大组合能力模型,毕竟。这种情况在现实中非常常见,比如Spring容器,如果将所有注解单独提出来,那么Spring本身可以看作是这种注解数据的解释器,容器的初始化便是这种数据解释执行的结果。
%
%又比如在魔兽世界等复杂游戏以及Photoshop这些大型绘图软件中,除了底层的高性能功能实现以外,通常会带有一个脚本语言,使得软件系统能够灵活配置。在魔兽世界中,一个NPC的对话,任务副本流程,这些功能相对于场景渲染,资源加载而言,首先他们容易发生变化,因此他们之间的隔离是必须的,其次相比于底层代码,这些数据的结构更加复杂,对性能的要求并不高。那么,可以将这种复杂系统看作是解释器,而底层模块则是这个解释器中的一个基本数据。

事实上,物理机器本身也可以看作是对机器指令的解释器。因此,解释器实际上是计算机系统中的一个通用结构。这种结构反映了计算机程序本身也是数据,而计算便是这种数据的解释执行的过程。

在语言本身的功能上,现代语言研究已经发展出许许多多的理论,在本论文中仅仅是对三个基本特性的组合与实现。在此基础上,诸如类型系统,更好的工具链,高性能的运行时,语言特性与并行,并发的结合都是相当精彩的研究方向。