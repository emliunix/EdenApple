%
% 第4章 parser
% 这个章节讲monad和parsec
% 毕竟monad也算是程序运行的一种抽象,还是有必要讲一讲的,
% 顺便把上次写的monad和parser的一点感想直接丢过来。
% 凑凑字数 <- 真实目的
%

% haskell中运算符的定义

\newcommand{\hsbind}[0]{$>\!\!>\!\!=$}
\newcommand{\hschoice}[0]{$<\!\!|\!\!>$}

\chapter{Monadic Parser Combinator}

对于一个完整的解释器而言,parser是必不可少的一部分。在上一章节中提及过EdenApple中的reader的唯一功能即parser。

通常,一个parser可以通过手写状态机,或者使用parser generator来生成,在这里,将描述一种以函数组合的形式来完成parse的方法,称之为parser combinator。

parser combinator作为一种parser编写的技巧并不怎么常见,这种技巧常见于函数式编程语言中,是一种将parser看作为函数并通过各种组合子一步步构建出复杂的解析规则的做法。一个比较出名的parser combinator的应用是Haskell中的parsec\cite{leijen01parsec}。

或许是因为函数式编程语言比较小众,而且parser combinator在处理左文法以及在复杂文法和性能上的缺点等等原因,在现实应用中,parser combinator不怎么流行。当然,在对parser combinator的研究上也一直不曾停歇。

\section{parser函数}

首先,从功能上来看,parser所完成的工作可以描述为,给定一个字符串,按照语言特定的语法规则解析这个字符串,最终返回一个解释结果,这个结果通常是一个树状结构的数据,反映了对应语法规则下的一个具体的语义结构。

如果将函数的定义与parser的工作过程对比,可以发现,一个parser可以看作是一个接收字符串为参数,返回一个解释结果的函数。

但是,仅仅是这种将parser看作是函数的看法并不能够带来什么实质性的去构建一个parser的方法。

在函数式编程中,一个重要的概念是高阶函数,一个函数可以接收函数作为参数,也可以返回一个函数。也就是说可以定义一类操作,将多个操作组合拼接成一个新的操作。parser combinator之所以称之为parser combinator,即是因为他的思想是将子parser按规则组合成为所需要的parser。

\section{组合子}

在描述组合子之前,先给出一些文法的例子以观察文法定义本身的组合特性。

首先,给出一个最简单的文法的例子:

\begin{listing}
\begin{verbatim}
<sample> ::= <chara><charb>
<chara>  ::= a
<charb>  ::= b
\end{verbatim}
\caption{顺序结构文法定义示例}
\label{lst:seq-syntax-sample}
\end{listing}

在\ref{lst:seq-syntax-sample}中,sample的定义由chara和charb两个自定义组成,解析sample的过程可以描述为先解析chara,然后解析charb。在这里,可以看到文法定义中的一种组合方式是顺序组合。

\begin{listing}
\begin{verbatim}
<sample> ::= <chara>|<charb>
<chara>  ::= a
<charb>  ::= b
\end{verbatim}
\caption{选择结构文法定义示例}
\label{lst:choice-syntax-sample}
\end{listing}

在示例\ref{lst:choice-syntax-sample}中,将\ref{lst:seq-syntax-sample}的定义稍微改变了一下,这里的意思是sample可能是<chara>,也可能是<charb>。在实际的解析过程中,可以先尝试将输入以chara的方式解析,如果失败,那么重新以charb的方式解析。

上面,介绍了词法规则中的两种组合方式,这两种组合方式基本上可以描述大部分情况的语法规则。同时,上面也提出了对于这两种文法组合方式的一种解析策略,需要注意的是这种解析策略并不适用所有可能的情况,比如左递归文法,如示例\ref{lst:ll-syntax-sample}中所示。

\begin{listing}
\begin{verbatim}
<expr> ::= <expr> + <expr> |
           <expr> - <expr> |
           <expr> * <expr> |
           <expr> / <expr> |
           (<expr>) |
           <number>
\end{verbatim}
\caption{左递归文法定义示例}
\label{lst:ll-syntax-sample}
\end{listing}

\section{形式化定义}

到目前为止,已经给出了将parser作为函数表示的思想,并通过示例文法给出了两种基本的组合方式,下面将以更加形式化的方式描述parser函数与组合子。

首先,给出parser函数的定义:\mintinline[breaklines]{haskell}{String -> (a, String)}。函数定义的标记规则采用Haskell规范。需要注意的是,parser函数返回的是一个自定义类型的数据和未解析字符串的二元组。因为解析过程通常不会消耗完所有的字符串,剩余未解析的部分需要返回作为下一步骤解析的输入。这个步骤通常不需要手动完成,下面介绍的bind组合子可以自动化这个过程。

解析器的操作对象是源码字符串,那么解析器组合子的操作对象就是解析器本身。通常可以看作是一个接收解析器为参数并返回一个组合后的解析器的函数。

在Monadic Parser Combinator中,与文法组合相对应的,有顺序组合的组合子,称之为bind(\hsbind)组合,还有一个选择操作的组合子,称之为choice(\hschoice)组合。

bind操作的签名为:\mintinline[breaklines]{haskell}{(>>=) :: m a -> (a -> m b) -> m b}。bind操作的定义为,返回一个m b类型的操作,该操作的行为是将m a操作的结果传入第二个参数所对应的函数并将这个函数所返回的操作的结果m b返回。

与此相对应的,\hschoice{}操作是一种平行组合。\hschoice{}操作将两个操作组合,首先将文本传入第一个操作,如果第一个操作失败,转而将相同的文本传入第二个操作。\hschoice{}的签名为\mintinline[breaklines]{haskell}{(<|>) :: m a -> m a -> m a}。

除了这两个基本组合外,还有一些常用的组合子和parser,加上这些基础元素,即可完成最基本的解析工作。这些基本元素有(定义P为 \mintinline[breaklines]{haskell}{type P a = String -> (a, String)}):

\begin{enumerate}
\item \mintinline{haskell}{char :: P Char} --- 解析一个字符并返回这个字符,
\item \mintinline{haskell}{return :: a -> P a}  --- 不消耗字符,直接返回结果,
\item \mintinline{haskell}{(>>) :: P a -> P b -> P b} --- 返回第二个解析器的结果。
\item \mintinline{haskell}{satisfy :: (Char -> Bool) -> P Char} --- 若下一个字符符合某种条件,则返回该字符
\item \mintinline{haskell}{many :: P a -> P [a]} --- 重复解析直到无法解析
\end{enumerate}

上述定义参考自monadic parser combinators\cite{hutton1996monadic}。

\section{Monad}

Monad是纯函数式编程中一个重要的概念,在Haskell中,依赖于Monad提供的代数结构组织具有顺序的一些列计算操作,比如IO,状态改变。这里提及两个Monad实现parser combinator所需的两个重要特性,第一个是上述的顺序性,parse作为一种计算操作需要按照一定的先后顺序执行,而且这种顺序也是文法定义的一部分。还一个是背景状态。在解析的过程中,解析器需要知道字符串正在解析的位置。在parser的定义中,可以看到返回的结果中包含了一个未解析的字符串,通过这个数据,在解析的过程中便可以知道当前解析操作已经进行到哪里了。而在组合的过程中,实际上并没有手动处理过这个数据,这是因为组合子在对parser进行组合的时候会自动的维护这个数据。比如在bind操作中,在完成第一个解析操作之后,会将结果中的这个\texttt{String}传入第二个解析操作。

需要注意的是,Monad只是一种组合子的风格,事实上也存在其他风格的组合子。在parsec论文\cite{leijen01parsec}中,描述了一种arrow style的组合子,并对两种风格做了简短的比较。其中arrow style的组合子的核心操作为:\mintinline[breaklines]{haskell}{(<*>) :: Parser a -> Parser (a -> b) -> Parser b}。arrow style的组合子的第二个操作与第一个操作完全独立,这导致了他只能表示context free的文法,而monad style的则可以表一定程度上的context sensitive的文法,比如图\ref{lst:xml-parser-sample}中的XML解析器。

\begin{code}
  \begin{minted}{haskell}
    xml = do{
      name <- openTag
      content <- many xml
      endTag name
      return (Node name content)
    } <|> xmlText
  \end{minted}
  \caption{XML Monad解析器片段}
  \label{lst:xml-parser-sample}
\end{code}

\section{Lisp文法}

Monadic parser combinator的优点便是写parser时语义能够直接表达出来,因此编写比较简单。在这一小节中,首先给出EdenApple中使用的Lisp文法,然后讨论实现上的细节。

Lisp使用S表达式来表达文法,S表达式只有两种基本结构,一个是原子(Atom),另一个是列表(List),其中一个特殊的是空列表或者null。对Atom和List的定义进行一些扩展就是所需要的Lisp文法。下面以ADT的形式给出LispVal的定义(以ADT而不是BNF给出定义一是因为ADT基本可以看作是BNF形式,还一个是因为ADT定义可以直接用以在程序中表示所需的解析结果数据类型):

\begin{listing}
\begin{verbatim}
LispVal = Symbol String |
          List [LispVal] |
          Number Integer |
          String String |
          Bool Bool
\end{verbatim}
\caption{LispVal定义}
\label{lst:lisp-val-def}
\end{listing}

实现上采用parsec的一个js实现