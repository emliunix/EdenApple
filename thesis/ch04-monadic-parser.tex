%
% 第4章 parser
% 这个章节讲monad和parsec
% 毕竟monad也算是程序运行的一种抽象,还是有必要讲一讲的,
% 顺便把上次写的monad和parser的一点感想直接丢过来。
% 凑凑字数 <- 真实目的
%

\chapter{Monadic Parser Combinator}

讲到Parser,这两天除了实现了一个基本的VM,剩下的时间基本上在鼓捣Parser。看了些许Parser Combinator的资料,重点在parsec上。parsec是一个Monadic Parser Combinator,在阅读过程中,略微有两点想法。
一个是对Monad的理解,Monad是一个代数结构,Monad的数据首先是一个函数(具备 * -> * 这样的kind),其次,对他有bind(>>=),>>,return,fail这几个操作。这几个操作的特点大概是都是以Monad数据为输入,返回的还是对应Monad的数据。
所以,如果把parser看作是一个接受String,返回结果和剩余String (:: (a, String))的函数,重点是把他看作函数,那么Parser Combinator就是对这些函数进行不断的组合。
Monad的组合操作都有个特点,就是结构上面体现了一种顺序结构。比如说bind操作,他的类型是 `m a -> (a -> m b) -> m b`。可以这么理解这个类型,首先进行`m a`操作,该操作的结果是`a`类型。再看第二个传入的值,他是一个函数,这个函数接受第一个操作的结果,并根据这个结果生成出第二个操作(类型为`m b`),第二个操作返回`b`类型的结果。那么这个先进行第一个操作,再进行第二个操作,并最终返回第二个操作的结果的着整个一个过程,也就是两个过程按顺序执行的组合过程,就是bind操作所生成的值。
第二个是关于Parser本身的,Parser有两个基本的组合,一个是上面的bind(`>>=`)组合,还一个choice(`<|>`)组合。在观察组合之前先观察parser本身,parser本身的类型 `String -> (a, String)`,反映了parser本身的功能,比较有意思的是返回的数据中还包括了未处理的文本数据。这样子可以在下一个parser传入这个文本,完成下一步的parse。正是这种结构,使得parser的组合变得简单直接。事实上,parser的bind操作就是这么一种操作:将第一个parse返回的未处理文本传入第二个parse。
由此来看,bind操作是顺序组合,与此相对应的,`<|>`操作是一种平行组合。`<|>`操作将两个操作组合,首先将文本传入第一个操作,如果第一个操作失败,转而将相同的文本传入第二个操作。
BNF文法中通常将一个元素表示为几个元素的顺序组合或者是用或`|`来表示两种不同的组合都是这个元素。比如:
```
<sexp> ::= <pair> | <atom>
<pair> ::= (<sexp> . <sexp>)
```
而这正好就是parsec中的两种基本组合。
monadic只是一种style,在parsec的论文中也讨论了另一种sytle,但是没仔细看,也不清楚是怎么实现组合的。关于parse,其实还有老多的东西不清楚,比如怎么稳妥的实现lookahead,怎么左递归,有哪些best practice什么的。
还有个就是在parse sexp的时候,感觉blank字符的处理还是比较麻烦,这种字符起到分割作用但又不是必须的,两个token之间不一定需要空白字符来分割。有一种想法是先lex一遍将字符串变成token串,然后在以parser转换成树状结构,再然后来一遍pass将这个树转换成Expr树,Expr树中会带上具体的语义,比如这是一棵Let树,那是一棵If树,还有Lambda树之类的。到这种状态,基本上对于compile而言就是万事俱备,只差临门一脚了。
