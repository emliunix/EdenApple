%
% 第一章 简介
%

\chapter{绪论}
\label{ch:intro}

Lisp 由人工智能之父John McCarthy发明,并于1960年发表于ACM\cite{mccarthy60}。LISP现在通常表示函数式编程语言中的一个族系,常见的LISP系语言有Common Lisp, Scheme, ELisp, Clojure。就像所有的函数式编程语言一样,Lisp以Lambda Calculus为理论基础,同时Lisp以S-Exp作为语法规则,因此带来的宏(Macro)功能能够让代码本身如同数据一样被操作。Lisp对后世许多编程语言造成了深刻影响。比如现今通用的词法作用域,垃圾回收机制,JS与Python中的闭包,等等。

Lisp语言用途广泛。Lisp在早期的AI相关研究中扮演重要的角色。Lisp因为本身结构简单,组合灵活,在教育上常用作一种教学用语言,同时,在工业上也有相当广泛的应用。曾应用于NASA,硅谷创业之父在他的《黑客与画家》中也描述了他创业时使用Lisp快速搭建出业务系统并因此击败竞争对手的故事。

在现阶段,Clojure作为一个新兴的Lisp系方言,在国内外也有着越来越多的使用案例。在国内有着LeanCloud, 一熊科技, 火币网等创业公司以Clojure作为主要开发语言。同时Clojure的JS版本ClojureScript也渐渐的在前端圈火热起来。

在通信等需要高可靠特性的环境下,通常使用Erlang\cite{armstrong07}来作为基础设施。而基于Erlang虚拟机的Erlang变体Elixir也深受Lisp的影响。同样受到Lisp影响的还有Julia,R,SPSS等。

Lisp的语法特性灵活,导致Lisp解释器或者编译器的实现往往异常困难。半个世纪以来有过许许多多的尝试去实现一个高效的Lisp运行环境。而这其中的关键,便是如何设计一个高效的运行时模型同时完整的支持Lisp的语义。

下面先讨论Lisp系语言中一些常见的语法特性,在后续的章节中具体讨论如何实现一个scheme解释器---EdenApple。该解释器由reader和VM组成。EdenApple的意义在于探索如何实现一个完整的解释器,一个完整的程序运行模型,因此,在本论文中,讨论的侧重点将置于解释器的VM的结构以及运行原理。
