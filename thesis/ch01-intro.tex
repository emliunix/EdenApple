%
% 第一章 简介
%
{
    \let\centering\raggedright
    \chapter*{绪论}
}
\addcontentsline{toc}{chapter}{\hspace{1em} 绪论}
\label{ch:intro}
\thispagestyle{hubu@thesis}

在本论文中,通过实现一个精简的Lisp解释器,研究如何实现Lisp这种动态类型函数式编程语言的运行时。论文首先探索了一个解释器的基本结构,再以几个基本的语言特性为目标探索了编程语言的运行时中的两种基本上下文,并且以最终实现一个可以运行的解释器来深入探索编程语言的运行行为。

编程语言的研究是计算机发展史上一个一直不曾停歇的主题,从最早期的指令输入,到汇编语言,到具有过程抽象的B语言,C语言,在到面向对象,函数式,多范式编程语言,以及具有复杂类型系统的Haskell,和用于定理证明的Coq,Idris等。编程语言从最早期的被忽视,到功能单一的抽象,如今已发展到百花齐放的时代。

现代的编程语言已经不再是一个可以简单的定义其功能的概念,作为具有一类形态的抽象工具,编程语言的内涵已经跨越逻辑学,科学计算,工程应用等多门学科。在当今社会,编程语言可以称之为社会基石的基石。

对于一个编程语言,程序如何定义,如何运行是一个基本的问题,而解答这个问题可以通过解释器这种抽象模型。一个解释器是运行一段计算机程序并计算得到最终结果的程序。研究编程语言的一个方向便是如何实现一个能够支持特定语言特性的解释器。

Lisp作为一个编程语言,可以看作是Lambda Calculus\cite{Church1932A}理论的一个直接翻译。Lambda Calculus是现代大多数函数式编程语言的理论基础,是与图灵机等价但在某些方面甚至比图灵机更加意义重大的一个理论。Lisp 由人工智能之父John McCarthy发明,并于1960年发表于ACM\cite{mccarthy60}。Lisp现在通常表示函数式编程语言中的一个族系,常见的Lisp系语言有Common Lisp, Scheme\cite{sussman1998scheme}, ELisp\cite{Lewis1993elisp}, Clojure\cite{hickey2008clj}。Lisp以S-Exp作为语法规则,因此带来的宏(Macro)功能能够让代码本身如同数据一样被操作。Lisp对后世许多编程语言造成了深刻影响。比如现今通用的词法作用域,垃圾回收机制,JS与Python中的闭包,等等。Lisp可以看作是编程语言发展历史中一个富有生命力,经久不衰的分支。

初代Lisp的具体内容可以参考\textit{LISP 1.5 programmer's manual}\cite{lisp1.5}。在该手册中详细讲述了Lisp的形式化定义,以及Lisp的语义与数据如何以S-Exp表达。

不同于解释执行。相对于编译器将一个编程语言转换为令一种语言,通常是线性的机器指令序列,解释执行通过一段现有的程序解析程序的语义结构并计算输入程序的结果。而无论是编译器还是解释执行,核心的一点便是对编程语言本身的解释,编译器只有在知道如何解释编程语言之后,才能够将对应的行为以另一种语言表达。而在本论文中,探索的便是这种解释的行为,具体的,采用了一种编译与解释执行相结合的解释流程。

对程序的解释当然离不开程序本身,因此,本论文虽然探索的是解释器的实现,但却离不开具体的语言特性,解释器的行为便是对具有这些语言特性的程序的解释。

在本论文中,所研究的解释器以Lisp作为具体语言,以闭包,call/cc,尾调用优化三个语言特性为目标,能够运行具备上述特性的Lisp程序。

\section*{国内外研究现状}

本论文研究的是Lisp语言的解释器,首先,Lisp语言本身用途广泛。Lisp在早期的AI相关研究中扮演重要的角色,事实上,正是因为AI领域的研究,Lisp才得以发明。Lisp因为本身结构简单,组合灵活,在教育上常用作一种教学用语言,同时,在工业上也有相当广泛的应用。曾应用于NASA。硅谷创业之父Paul Graham在他的《黑客与画家》中也描述了他创业时使用Lisp快速搭建出业务系统并因此击败竞争对手的故事。

Lisp方言以及对应方言解释器的探索一直不曾终止。上个世纪便有着Common Lisp,ELisp,scheme等各具特色的流行变体。在近些年,亦有Lisp的变体出现,比如Clojure,Clojure作为一个新兴的Lisp系方言,在国内外也有着越来越多的使用案例。在国内有着LeanCloud, 一熊科技, 火币网等创业公司以Clojure作为主要开发语言。同时Clojure的JS版本ClojureScript也渐渐的在前端圈火热起来。

在通信等需要高可靠特性的环境下,通常使用Erlang\cite{armstrong07}来作为基础设施。而基于Erlang虚拟机的Erlang变体Elixir也深受Lisp的影响。同样受到Lisp影响的还有Julia,R,SPSS等。

Racket\cite{plt-tr2010-ref}项目是一个scheme的超集,由多个大学与企业主导,在其主页中,将Racket描述成是一个新语言特性的研究场。这表明在现阶段,在有关语言特性的研究中,Lisp仍然占据一定的分量,与Haskell,Ocaml等语言都是语言研究中的主要工具。

可以看到Lisp(包括其灵活的文法,完善的基本语言特性)可以看作是语言研究的一个非常完备的原型,在其上可以方便的演变出各种语言特性。

除了Lisp系语言的发展以外,现阶段编程语言的研究一个是工程应用上追求效率与正确性的努力,这方面成就有近些年的Swift,Rust等,也有走向完善类型系统的方向的Haskell系,ML系语言。而每一个方向的研究都离不开对语言运行环境的设计以使其能够高效运行。可以说,这些语言的大部分研究重点在于运行环境,或者说是``解释器''的设计。

在这方面的成果有LLVM(一个编译器框架),G-Machine(一个函数式语言的运行环境)等等。

\section*{研究内容}
%\addcontentsline{toc}{section}{研究内容}

本论文研究如何实现一个具备词法作用域,call/cc,尾调用优化这三种基本语言特性的基础Lisp解释器以及该语言运行时的模型。为讨论方便,将其命名为EdenApple,以表示其与其他解释器或编译器的区别。EdenApple采用scheme作为参考语言。

在此选择Lisp作为解释器实现的目标语言一个是因为Lisp族系语言本身在语法特性上高度统一并且丰富,其次,Lisp的文法(S Expression)也以其简单强大的表达能力闻名,这使得研究过程中可以专注于语言特性本身。

Lisp的语法特性灵活,导致Lisp解释器或者编译器的实现往往异常困难。半个世纪以来有过许许多多的尝试去实现一个高效的Lisp运行环境。而这其中的关键,便是如何设计一个高效的运行时模型同时完整的支持Lisp的语义。

这些尝试带来了各种不同的Lisp实现,比如T语言以及其Orbit编译器\cite{Adams1986ORBIT},Chez Scheme\cite{dybvig2006chez},等等,他们在运行机制,编译目标,运行效率等等方面各有千秋。本论文中的解释器由reader和VM组成,是一种基于堆的解释器模型。而对其实现,虽然不限置于任何语言,本论文选择JavaScript作为运行的宿主环境。

EdenApple的意义在于探索如何实现一个完整的程序运行模型,因此,在本论文中,讨论的侧重点在于解释器的VM的结构以及运行原理。VM的结构主要基于Kent. R Dybvig的\textit{Three implementation models for scheme\cite{dybvig87timpl}},其中letrec的实现主要参考Kent的另一篇文章\textit{Fixing letrec\cite{Waddell2005Dybvig}}。

\section*{论文组织}
%\addcontentsline{toc}{section}{论文组织}

本论文主要分为两部分,前半部分讨论语言特性与解释器结构,后半部分讨论EdenApple的实现。

在\nameref{ch:lang features}中讨论EdenApple的三个基本语法特性,在\nameref{ch:interp structure}中描述解释器的基本结构,随后在\nameref{ch:parser}中讨论如何实现解释器的解析器部分,解析器的实现采用了parser combinator技术。最后在\nameref{ch:vm impl}中详细讨论EdenApple的虚拟机模型以及该模型如何支持语言的运行,具体的,如何实现上述的三个基本语言特性。